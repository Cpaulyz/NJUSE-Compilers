% hw2.tex

% !TEX program = xelatex
%%%%%%%%%%%%%%%%%%%%
% see http://mirrors.concertpass.com/tex-archive/macros/latex/contrib/tufte-latex/sample-handout.pdf
% for how to use tufte-handout
\documentclass[a4paper, justified]{tufte-handout}

\input{hw-preamble} % feel free to modify this file if you understand LaTeX well
%%%%%%%%%%%%%%%%%%%%
\title{编译原理作业 (2)}
\me{陈彦泽}{181250015}{}{}
\date{2020年11月19日}
%%%%%%%%%%%%%%%%%%%%
\begin{document}
\maketitle
%%%%%%%%%%%%%%%%%%%%
\noplagiarism % PLEASE DON'T DELETE THIS LINE!
%%%%%%%%%%%%%%%%%%%%
\begin{abstract}
  % \mfigcap{width = 0.85\textwidth}{figs/George-Boole}{George Boole}
  % \begin{center}{\fcolorbox{blue}{yellow!60}{\parbox{0.65\textwidth}{\large 
  %   \begin{itemize}
  %     \item 
  %   \end{itemize}}}}
  % \end{center}
\end{abstract}
%%%%%%%%%%%%%%%%%%%%
\beginrequired
%%%%%%%%%%%%%%%

%%%%%%%%%%%%%%%
\begin{problem}[正则表达式与自动机 \score{10 = 2 + 2 + 2 + 2 + 2}]
  考虑正则表达式 $r = a(b|c)^{\ast}$。
  \sidenote{
    如何用 \LaTeX{} 写(复杂的)正则表达式?
    \begin{itemize}
      \item \href{https://tex.stackexchange.com/a/162122/23098}{How to escape properly and output regex in latex?@tex.stackexchange}
    \end{itemize}
  }
  \begin{enumerate}[(1)]
    \item 使用 Thompson 构造法构造等价的 NFA; \sidenote{
      如何用 \LaTeX{} 画自动机?
      \begin{itemize}
        \item \href{https://www3.nd.edu/~kogge/courses/cse30151-fa17/Public/other/tikz\_tutorial.pdf}{使用{\texttt{tikz automata}} library}
        \item \href{https://hayesall.com/blog/latex-automata/}{另一个关于\texttt{tikz automata}的教程}
        \item 在\href{https://notendur.hi.is/aee11/automataLatexGen/}{网站\texttt{automataLatexGen}生成\LaTeX{}代码}
      \end{itemize}}
    \item 使用子集构造法构造等价的 DFA;
    \item 将上一步构造的 DFA 最小化;
    \item 将上一步得到的最小 DFA 转化为等价的正则表达式, 记为 $r'$。
    \item $r'$ 与 $r$ 相同吗? 如果不同, 请将 $r'$ 化简为 $r$。
  \end{enumerate}
  以上各小题, 请给出关键的中间步骤。
  
  \noindent (不必给出所有的细节, 类似的步骤可以``跳步''; 尽量将解答部分控制在两页以内。)

\end{problem}

\begin{solution}
  (1)\\

  \fig{width = 0.80\textwidth}{figs/nfa}
  (2)

  A=\{0\}

  A$\stackrel{a}{\longrightarrow}$B=\{1、2、3、5、8\}

  B$\stackrel{b}{\longrightarrow}$C=\{2、3、5、6、7、8\}

  B$\stackrel{c}{\longrightarrow}$ D=\{2、3、4、5、7、8\}

  C$\stackrel{b}{\longrightarrow}$ C, C$\stackrel{c}{\longrightarrow}$ D, D$\stackrel{b}{\longrightarrow}$ C, D$\stackrel{c}{\longrightarrow}$D
  \fig{width = 0.50\textwidth}{figs/dfa}
  
  (3)\\

  $\Pi_0$ = \{\{A\},\{B,C,D\}\}

  $\Pi_1$=$\Pi_0$,不可再分,说明$\Pi_0$已经是最小化DFA
  \fig{width = 0.50\textwidth}{figs/nfa2}

  (4)\\

  \textbf{init}

  $R_{00}^{-1} = \epsilon $  

  $R_{01}^{-1} = a$

  $R_{10}^{-1} = \varnothing$

  $R_{11}^{-1} = b|c|\epsilon$\\

  \textbf{step0}

  $R_{00}^{0} = R_{00}^{-1}(R_{00}^{-1})^*R_{00}^{-1} | R_{00}^{-1} = \epsilon(\epsilon)^*\epsilon|\epsilon = \epsilon $

  $R_{01}^{0} = R_{00}^{-1}(R_{00}^{-1})^*R_{01}^{-1} | R_{01}^{-1} = \epsilon(\epsilon)^*a|a = a $

  $R_{10}^{0} = R_{10}^{-1}(R_{00}^{-1})^*R_{00}^{-1} | R_{10}^{-1} = \varnothing(\epsilon)^*\epsilon|\varnothing = \varnothing $

  $R_{11}^{0} = R_{10}^{-1}(R_{00}^{-1})^*R_{01}^{-1} | R_{11}^{-1} = \varnothing(\epsilon)^*a|(b|c|\epsilon) = b|c|\epsilon $\\

  \textbf{step1}

  $R_{01}^{0} = R_{01}^{0}(R_{11}^{0})^*R_{11}^{0} | R_{01}^{0} = a(b|c|\epsilon)^*(b|c|\epsilon)|a = a(b|c|\epsilon)+=a(b|c)^* $\\

  \textbf{result}

  $r' = a(b|c)^*$

  (5)\\
  
  相同


\end{solution}

%%%%%%%%%%%%%%%

%%%%%%%%%%%%%%%%%%%%
% 如果没有需要订正的题目,可以把这部分删掉

% \begincorrection
%%%%%%%%%%%%%%%%%%%%

%%%%%%%%%%%%%%%%%%%%
% 如果没有反馈,可以把这部分删掉
% \beginfb

% 你可以写
% \begin{itemize}
%   \item 对课程及教师的建议与意见
%   \item 教材中不理解的内容
%   \item 希望深入了解的内容
%   \item $\cdots$
% \end{itemize}
% %%%%%%%%%%%%%%%%%%%%
\end{document}