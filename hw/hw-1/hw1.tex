% hw1.tex

% !TEX program = xelatex
%%%%%%%%%%%%%%%%%%%%
% see http://mirrors.concertpass.com/tex-archive/macros/latex/contrib/tufte-latex/sample-handout.pdf
% for how to use tufte-handout
\documentclass[a4paper, justified]{tufte-handout}

\input{hw-preamble} % feel free to modify this file if you understand LaTeX well
%%%%%%%%%%%%%%%%%%%%
\title{编译原理作业 (1)}
\me{陈彦泽}{181250015}{}{}
\date{\zhtoday}
%%%%%%%%%%%%%%%%%%%%
\begin{document}
\maketitle
%%%%%%%%%%%%%%%%%%%%
\noplagiarism % PLEASE DON'T DELETE THIS LINE!
%%%%%%%%%%%%%%%%%%%%
\begin{abstract}
  % \mfigcap{width = 0.85\textwidth}{figs/George-Boole}{George Boole}
  % \begin{center}{\fcolorbox{blue}{yellow!60}{\parbox{0.65\textwidth}{\large 
  %   \begin{itemize}
  %     \item 
  %   \end{itemize}}}}
  % \end{center}
\end{abstract}
%%%%%%%%%%%%%%%%%%%%
\beginrequired
%%%%%%%%%%%%%%%
\begin{problem}[编译器, 然后呢?]
  观看系列视频 \href{https://www.bilibili.com/video/BV1EW411u7th?}{Crash Course Computer Science@bilibili}:
  \begin{itemize}
    \item $P5 \sim P8$; 总时长约45分钟
    \item 目的: 了解机器语言是如何跑起来的
    \item {\bf 作业:} 随便写点什么吧 (要能表明你确实学习了这些视频)
  \end{itemize}
\end{problem}

\begin{solution}
  “根本上,这些技术都是矩阵层层嵌套,来存储大量信息,就像计算机中的很多事情,底层其实很简单,让人难以理解的是一层层精妙的抽象,像一个越来越小的俄罗斯套娃”\\

  这组视频用了不到一个小时的时间,用很动画的方式复习了一遍计组的内容。实际上里面的思想用上述的一句话就可以概括\\
  
  从ALU说起,ALU本质上就是逻辑门的抽象,用最基本的AND、OR、NOT、XOR电路组成了一个一位的加法器,再用8个一位的加法器(举个例子)组成了八位的全加器,
  最后把各组运算单元进行总体抽象,抽象成一个ALU,接受两个输入InputA、InputB和操作码OpCode,输出Ouput并设置FLAG。\\

  再到存储,寄存器就是一组锁存器的矩阵抽象,而锁存器则是利用逻辑门组成门锁的一个抽象。\\

  最后是CPU,CPU根据时钟周期自动地从内存中取指令、解码、执行的过程利用到了先前抽象的ALU,与存储的交互中直接与存储抽象的地址线和使能线进行交互,
  可以看作是对ALU和寄存器、时钟等的一个抽象。

\end{solution}
%%%%%%%%%%%%%%%

%%%%%%%%%%%%%%%
\begin{problem}[手写词法分析器]
  根据下面的状态转移图以及课上介绍的识别方法, 给出识别数字
  (正整数、不带科学计数法的浮点数以及带科学计数法的浮点数)的伪代码。
  (推荐使用 \LaTeX{} 
  \href{http://tug.ctan.org/macros/latex/contrib/algorithmicx/algorithmicx.pdf}{\texttt{algorithmicx} 包}
  书写伪代码)

  \fig{width = 0.80\textwidth}{figs/number}

  例如, 对于输入串 \texttt{1.23E+a4.5E6b78.c}, 
  应该识别出 \texttt{1.23}, \texttt{4.5E6}, \texttt{78},
  并且其余字符均应被视为神秘的未知字符。
\end{problem}

\begin{solution}
\end{solution}

\begin{algorithm}
  \caption{识别数字算法} 
  \begin{algorithmic}[1]
    \Function{readInt}{}\Comment{定义重复读digit转为整数的函数,供复用}
      \State $v\gets 0$
      \While{isDight($peek$)}
        \State $v\gets 10*v+toInt(peek)$
        \State $peek \gets read()$
      \EndWhile
      \State  \textbf{return} $v$
    \EndFunction
    \Procedure{start}{} \Comment{start}
      \State $left \gets readInt()$  \Comment{循环读取digit到状态13}
      \If{$peek$='.'}                 \Comment{如果有.,读小数部分}
        \State $peek \gets read()$
        \State $r \gets readInt()$
      \ElsIf{$peek\not=$'E'}
        \State  \textbf{return} $INT(l)$ \Comment{other 20}
      \EndIf
      \If{$peek$='E'}\Comment{读到E,接下来读指数部分,状态16}
        \State $peek \gets read()$
        \If{$peek$='-'}\Comment{如果读到了-,将符号位neg记为1}
          \State $peek \gets read()$
          \State $neg \gets 1$
        \EndIf
        \State $e \gets readInt()$ \Comment{循环读取digit到状态13}
        \If{$neg$='1'}
        \State $e \gets -e$ \Comment{如果符号位为1,取负}
        \EndIf
        \State \textbf{return} $FLOAT(l+r/len(r)*pow(10,e))$ \Comment{other 19}
      \ElsIf{$peek\not=$'E'}
        \State \textbf{return} $FLOAT(l+r/len(r))$ \Comment{other 21}
      \EndIf
    \EndProcedure
  \end{algorithmic}
\end{algorithm}
%%%%%%%%%%%%%%%

%%%%%%%%%%%%%%%
\begin{problem}[正则表达式]
  课堂上, 我们提到了下面的正则表达式可以用于识别所有(二进制表示的)3的倍数。
  请证明该结论(或给出直观的解释)。
  参考: \teal{\url{https://regex101.com/r/ED4qgC/1}}
  \[ 
    \Big(0|\big(1(01^{\ast}0)^{\ast}1\big)\Big)^{\ast} 
  \]
\end{problem}

\begin{solution}
  该正则表达式可以通过有限状态自动机进行解释。\\\\
  因为mod3的情况下,余数的状态有0、1、2三种。设定三个状态,分别叫做0、1和2,它们表示当前的数除以3所得的余数。\\\\
  因为二进制数左移一位表示十进制中的乘2,我们从左往右读二进制串,每读一位0或者1进行一次状态转移,即
  如果对于某个i和j,有i*2≡j (mod 3),就加一条路径i→j,路径上标一个字符“0”或者“1”。\\\\
  例如,二进制串1011,我们从左往右读,初始状态为0,则读入1后状态为1,表示余数为1;读入0后状态为2,表示余数为2;
  读入1后状态为1,表示余数为1;读入1后状态为1,表示余数为1。\\\\
  构建出的有限状态自动机如图所示:
  \fig{width = 0.50\textwidth}{figs/figure}\\
  我们的目标是识别3的倍数,即mod3余0,也就是表示状态为0的所有可能,既0为接受状态。\\\\
  我们简单枚举从0出发,回到0的路径可能:
  \begin{itemize}
    \item 从0到0:0
    \item 从0到1到0:11
    \item 从0到1到2到1到0:1001
    \item 因为在状态2下可以循环,改为:10(1*)01
    \item 因为在状态12间可以循环,改为1(0(1*)0)*1
  \end{itemize}

  根据Kleene构造法,表示为正则表达式,即为:
  \[ 
    \Big(0|\big(1(01^{\ast}0)^{\ast}1\big)\Big)^{\ast} 
  \]

\end{solution}
%%%%%%%%%%%%%%%

%%%%%%%%%%%%%%%%%%%%
% 如果没有需要订正的题目,可以把这部分删掉

% \begincorrection
%%%%%%%%%%%%%%%%%%%%

%%%%%%%%%%%%%%%%%%%%
% 如果没有反馈,可以把这部分删掉
\beginfb


\begin{itemize}
  \item 老师真的很用心上课,爱了爱了!
\end{itemize}
%%%%%%%%%%%%%%%%%%%%
\end{document}