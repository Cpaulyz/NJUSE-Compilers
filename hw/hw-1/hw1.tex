% hw1.tex

% !TEX program = xelatex
%%%%%%%%%%%%%%%%%%%%
% see http://mirrors.concertpass.com/tex-archive/macros/latex/contrib/tufte-latex/sample-handout.pdf
% for how to use tufte-handout
\documentclass[a4paper, justified]{tufte-handout}

% hw-preamble.tex

% geometry for A4 paper
% See https://tex.stackexchange.com/a/119912/23098
\geometry{
  left=20.0mm,
  top=20.0mm,
  bottom=20.0mm,
  textwidth=130mm, % main text block
  marginparsep=5.0mm, % gutter between main text block and margin notes
  marginparwidth=50.0mm % width of margin notes
}

% for colors
\usepackage{xcolor} % usage: \color{red}{text}
% predefined colors
\newcommand{\red}[1]{\textcolor{red}{#1}} % usage: \red{text}
\newcommand{\blue}[1]{\textcolor{blue}{#1}}
\newcommand{\teal}[1]{\textcolor{teal}{#1}}

\usepackage{todonotes}

% heading
\usepackage{sectsty}
\setcounter{secnumdepth}{2}
\allsectionsfont{\centering\huge\rmfamily}

% for Chinese
\usepackage{xeCJK}
\usepackage{zhnumber}
\setCJKmainfont[BoldFont=FandolSong-Bold.otf]{FandolSong-Regular.otf}

% for fonts
\usepackage{fontspec}
\newcommand{\song}{\CJKfamily{song}}
\newcommand{\kai}{\CJKfamily{kai}}

% To fix the ``MakeTextLowerCase'' bug:
% See https://github.com/Tufte-LaTeX/tufte-latex/issues/64#issuecomment-78572017
% Set up the spacing using fontspec features
\renewcommand\allcapsspacing[1]{{\addfontfeature{LetterSpace=15}#1}}
\renewcommand\smallcapsspacing[1]{{\addfontfeature{LetterSpace=10}#1}}

% for url
\usepackage{hyperref}
\hypersetup{colorlinks = true,
  linkcolor = teal,
  urlcolor  = teal,
  citecolor = blue,
  anchorcolor = blue}

\newcommand{\me}[4]{
    \author{
      {\bfseries 姓名:}\underline{#1}\hspace{2em}
      {\bfseries 学号:}\underline{#2}\hspace{2em}\\[10pt]
      {\bfseries 评分:}\underline{#3\hspace{3em}}\hspace{2em}
      {\bfseries 评阅:}\underline{#4\hspace{3em}}
  }
}

% Please ALWAYS Keep This.
\newcommand{\noplagiarism}{
  \begin{center}
    \fbox{\begin{tabular}{@{}c@{}}
      请独立完成作业,不得抄袭。\\
      若得到他人帮助, 请致谢。\\
      若参考了其它资料,请给出引用。\\
      鼓励讨论,但需独立书写解题过程。
    \end{tabular}}
  \end{center}
}

% \newcommand{\goal}[1]{
%   \begin{center}{\fcolorbox{blue}{yellow!60}{\parbox{0.50\textwidth}{\large
%     \begin{itemize}
%       \item 体会``思维的乐趣''
%       \item 初步了解递归与数学归纳法
%       \item 初步接触算法概念与问题下界概念
%     \end{itemize}}}}
%   \end{center}
% }

% Each hw consists of four parts:
\newcommand{\beginrequired}{\hspace{5em}\section{作业 (必做部分)}}
\newcommand{\beginoptional}{\section{作业 (选做部分)}}
\newcommand{\beginot}{\section{Open Topics}}
\newcommand{\begincorrection}{\section{订正}}
\newcommand{\beginfb}{\section{反馈}}

% for math
\usepackage{amsmath, mathtools, amsfonts, amssymb}
\newcommand{\set}[1]{\{#1\}}

% define theorem-like environments
\usepackage[amsmath, thmmarks]{ntheorem}

\theoremstyle{break}
\theorempreskip{2.0\topsep}
\theorembodyfont{\song}
\theoremseparator{}
\newtheorem{problem}{题目}[subsection]
\renewcommand{\theproblem}{\arabic{problem}}
\newtheorem{ot}{Open Topics}

\theorempreskip{3.0\topsep}
\theoremheaderfont{\kai\bfseries}
\theoremseparator{:}
\theorempostwork{\bigskip\hrule}
\newtheorem*{solution}{解答}
\theorempostwork{\bigskip\hrule}
\newtheorem*{revision}{订正}

\theoremstyle{plain}
\newtheorem*{cause}{错因分析}
\newtheorem*{remark}{注}

\theoremstyle{break}
\theorempostwork{\bigskip\hrule}
\theoremsymbol{\ensuremath{\Box}}
\newtheorem*{proof}{证明}

% \newcommand{\ot}{\blue{\bf [OT]}}

% for figs
\renewcommand\figurename{图}
\renewcommand\tablename{表}

% for fig without caption: #1: width/size; #2: fig file
\newcommand{\fig}[2]{
  \begin{figure}[htbp]
    \centering
    \includegraphics[#1]{#2}
  \end{figure}
}
% for fig with caption: #1: width/size; #2: fig file; #3: caption
\newcommand{\figcap}[3]{
  \begin{figure}[htbp]
    \centering
    \includegraphics[#1]{#2}
    \caption{#3}
  \end{figure}
}
% for fig with both caption and label: #1: width/size; #2: fig file; #3: caption; #4: label
\newcommand{\figcaplbl}[4]{
  \begin{figure}[htbp]
    \centering
    \includegraphics[#1]{#2}
    \caption{#3}
    \label{#4}
  \end{figure}
}
% for margin fig without caption: #1: width/size; #2: fig file
\newcommand{\mfig}[2]{
  \begin{marginfigure}
    \centering
    \includegraphics[#1]{#2}
  \end{marginfigure}
}
% for margin fig with caption: #1: width/size; #2: fig file; #3: caption
\newcommand{\mfigcap}[3]{
  \begin{marginfigure}
    \centering
    \includegraphics[#1]{#2}
    \caption{#3}
  \end{marginfigure}
}

\usepackage{fancyvrb}

% for algorithms
\usepackage[]{algorithm}
\usepackage[]{algpseudocode} % noend
% See [Adjust the indentation whithin the algorithmicx-package when a line is broken](https://tex.stackexchange.com/a/68540/23098)
\newcommand{\algparbox}[1]{\parbox[t]{\dimexpr\linewidth-\algorithmicindent}{#1\strut}}
\newcommand{\hStatex}[0]{\vspace{5pt}}
\makeatletter
\newlength{\trianglerightwidth}
\settowidth{\trianglerightwidth}{$\triangleright$~}
\algnewcommand{\LineComment}[1]{\Statex \hskip\ALG@thistlm \(\triangleright\) #1}
\algnewcommand{\LineCommentCont}[1]{\Statex \hskip\ALG@thistlm%
  \parbox[t]{\dimexpr\linewidth-\ALG@thistlm}{\hangindent=\trianglerightwidth \hangafter=1 \strut$\triangleright$ #1\strut}}
\makeatother

% for footnote/marginnote
% see https://tex.stackexchange.com/a/133265/23098
\usepackage{tikz}
\newcommand{\circled}[1]{%
  \tikz[baseline=(char.base)]
  \node [draw, circle, inner sep = 0.5pt, font = \tiny, minimum size = 8pt] (char) {#1};
}
\renewcommand\thefootnote{\protect\circled{\arabic{footnote}}}

\newcommand{\score}[1]{{\bf [#1 分]}}

\newcommand{\rel}[1]{\xrightarrow{#1}}
\newcommand{\dstar}{\xRightarrow[]{\ast}}
\newcommand{\dplus}{\xRightarrow[]{+}}
\newcommand{\lm}{\xRightarrow[\text{lm}]{}}
\renewcommand{\rm}{\xRightarrow[\text{rm}]{}}
\newcommand{\dpluslm}{\xRightarrow[\text{lm}]{+}}
\newcommand{\dstarlm}{\xRightarrow[\text{lm}]{\ast}}
\newcommand{\dplusrm}{\xRightarrow[\text{rm}]{+}}
\newcommand{\dstarrm}{\xRightarrow[\text{rm}]{\ast}}

\newcommand{\first}{\textsc{First}}
\newcommand{\follow}{\textsc{Follow}}

% see https://tex.stackexchange.com/a/109906/23098
\usepackage{empheq}
\newcommand*\widefbox[1]{\fbox{\hspace{2em}#1\hspace{2em}}}


\usepackage{multirow} % feel free to modify this file if you understand LaTeX well
%%%%%%%%%%%%%%%%%%%%
\title{编译原理作业 (1)}
\me{陈彦泽}{181250015}{}{}
\date{\zhtoday}
%%%%%%%%%%%%%%%%%%%%
\begin{document}
\maketitle
%%%%%%%%%%%%%%%%%%%%
\noplagiarism % PLEASE DON'T DELETE THIS LINE!
%%%%%%%%%%%%%%%%%%%%
\begin{abstract}
  % \mfigcap{width = 0.85\textwidth}{figs/George-Boole}{George Boole}
  % \begin{center}{\fcolorbox{blue}{yellow!60}{\parbox{0.65\textwidth}{\large 
  %   \begin{itemize}
  %     \item 
  %   \end{itemize}}}}
  % \end{center}
\end{abstract}
%%%%%%%%%%%%%%%%%%%%
\beginrequired
%%%%%%%%%%%%%%%
\begin{problem}[编译器, 然后呢?]
  观看系列视频 \href{https://www.bilibili.com/video/BV1EW411u7th?}{Crash Course Computer Science@bilibili}:
  \begin{itemize}
    \item $P5 \sim P8$; 总时长约45分钟
    \item 目的: 了解机器语言是如何跑起来的
    \item {\bf 作业:} 随便写点什么吧 (要能表明你确实学习了这些视频)
  \end{itemize}
\end{problem}

\begin{solution}
  “根本上,这些技术都是矩阵层层嵌套,来存储大量信息,就像计算机中的很多事情,底层其实很简单,让人难以理解的是一层层精妙的抽象,像一个越来越小的俄罗斯套娃”\\

  这组视频用了不到一个小时的时间,用很动画的方式复习了一遍计组的内容。实际上里面的思想用上述的一句话就可以概括\\
  
  从ALU说起,ALU本质上就是逻辑门的抽象,用最基本的AND、OR、NOT、XOR电路组成了一个一位的加法器,再用8个一位的加法器(举个例子)组成了八位的全加器,
  最后把各组运算单元进行总体抽象,抽象成一个ALU,接受两个输入InputA、InputB和操作码OpCode,输出Ouput并设置FLAG。\\

  再到存储,寄存器就是一组锁存器的矩阵抽象,而锁存器则是利用逻辑门组成门锁的一个抽象。\\

  最后是CPU,CPU根据时钟周期自动地从内存中取指令、解码、执行的过程利用到了先前抽象的ALU,与存储的交互中直接与存储抽象的地址线和使能线进行交互,
  可以看作是对ALU和寄存器、时钟等的一个抽象。

\end{solution}
%%%%%%%%%%%%%%%

%%%%%%%%%%%%%%%
\begin{problem}[手写词法分析器]
  根据下面的状态转移图以及课上介绍的识别方法, 给出识别数字
  (正整数、不带科学计数法的浮点数以及带科学计数法的浮点数)的伪代码。
  (推荐使用 \LaTeX{} 
  \href{http://tug.ctan.org/macros/latex/contrib/algorithmicx/algorithmicx.pdf}{\texttt{algorithmicx} 包}
  书写伪代码)

  \fig{width = 0.80\textwidth}{figs/number}

  例如, 对于输入串 \texttt{1.23E+a4.5E6b78.c}, 
  应该识别出 \texttt{1.23}, \texttt{4.5E6}, \texttt{78},
  并且其余字符均应被视为神秘的未知字符。
\end{problem}

\begin{solution}
\end{solution}

\begin{algorithm}
  \caption{识别数字算法} 
  \begin{algorithmic}[1]
    \Function{readInt}{}\Comment{定义重复读digit转为整数的函数,供复用}
      \State $v\gets 0$
      \While{isDight($peek$)}
        \State $v\gets 10*v+toInt(peek)$
        \State $peek \gets read()$
      \EndWhile
      \State  \textbf{return} $v$
    \EndFunction
    \Procedure{start}{} \Comment{start}
      \State $left \gets readInt()$  \Comment{循环读取digit到状态13}
      \If{$peek$='.'}                 \Comment{如果有.,读小数部分}
        \State $peek \gets read()$
        \State $r \gets readInt()$
      \ElsIf{$peek\not=$'E'}
        \State  \textbf{return} $INT(l)$ \Comment{other 20}
      \EndIf
      \If{$peek$='E'}\Comment{读到E,接下来读指数部分,状态16}
        \State $peek \gets read()$
        \If{$peek$='-'}\Comment{如果读到了-,将符号位neg记为1}
          \State $peek \gets read()$
          \State $neg \gets 1$
        \EndIf
        \State $e \gets readInt()$ \Comment{循环读取digit到状态13}
        \If{$neg$='1'}
        \State $e \gets -e$ \Comment{如果符号位为1,取负}
        \EndIf
        \State \textbf{return} $FLOAT(l+r/len(r)*pow(10,e))$ \Comment{other 19}
      \ElsIf{$peek\not=$'E'}
        \State \textbf{return} $FLOAT(l+r/len(r))$ \Comment{other 21}
      \EndIf
    \EndProcedure
  \end{algorithmic}
\end{algorithm}
%%%%%%%%%%%%%%%

%%%%%%%%%%%%%%%
\begin{problem}[正则表达式]
  课堂上, 我们提到了下面的正则表达式可以用于识别所有(二进制表示的)3的倍数。
  请证明该结论(或给出直观的解释)。
  参考: \teal{\url{https://regex101.com/r/ED4qgC/1}}
  \[ 
    \Big(0|\big(1(01^{\ast}0)^{\ast}1\big)\Big)^{\ast} 
  \]
\end{problem}

\begin{solution}
  该正则表达式可以通过有限状态自动机进行解释。\\\\
  因为mod3的情况下,余数的状态有0、1、2三种。设定三个状态,分别叫做0、1和2,它们表示当前的数除以3所得的余数。\\\\
  因为二进制数左移一位表示十进制中的乘2,我们从左往右读二进制串,每读一位0或者1进行一次状态转移,即
  如果对于某个i和j,有i*2≡j (mod 3),就加一条路径i→j,路径上标一个字符“0”或者“1”。\\\\
  例如,二进制串1011,我们从左往右读,初始状态为0,则读入1后状态为1,表示余数为1;读入0后状态为2,表示余数为2;
  读入1后状态为1,表示余数为1;读入1后状态为1,表示余数为1。\\\\
  构建出的有限状态自动机如图所示:
  \fig{width = 0.50\textwidth}{figs/figure}\\
  我们的目标是识别3的倍数,即mod3余0,也就是表示状态为0的所有可能,既0为接受状态。\\\\
  我们简单枚举从0出发,回到0的路径可能:
  \begin{itemize}
    \item 从0到0:0
    \item 从0到1到0:11
    \item 从0到1到2到1到0:1001
    \item 因为在状态2下可以循环,改为:10(1*)01
    \item 因为在状态12间可以循环,改为1(0(1*)0)*1
  \end{itemize}

  根据Kleene构造法,表示为正则表达式,即为:
  \[ 
    \Big(0|\big(1(01^{\ast}0)^{\ast}1\big)\Big)^{\ast} 
  \]

\end{solution}
%%%%%%%%%%%%%%%

%%%%%%%%%%%%%%%%%%%%
% 如果没有需要订正的题目,可以把这部分删掉

% \begincorrection
%%%%%%%%%%%%%%%%%%%%

%%%%%%%%%%%%%%%%%%%%
% 如果没有反馈,可以把这部分删掉
\beginfb


\begin{itemize}
  \item 老师真的很用心上课,爱了爱了!
\end{itemize}
%%%%%%%%%%%%%%%%%%%%
\end{document}