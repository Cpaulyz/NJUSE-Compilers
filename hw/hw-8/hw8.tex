% hw8.tex

% !TEX program = xelatex
%%%%%%%%%%%%%%%%%%%%
% see http://mirrors.concertpass.com/tex-archive/macros/latex/contrib/tufte-latex/sample-handout.pdf
% for how to use tufte-handout
\documentclass[a4paper, justified]{tufte-handout}

\input{hw-preamble} % feel free to modify this file if you understand LaTeX well
%%%%%%%%%%%%%%%%%%%%
\title{编译原理作业 (8)}
\me{陈彦泽}{181250015}{}{}
\date{2020年12月31日}
%%%%%%%%%%%%%%%%%%%%
\begin{document}
\maketitle
%%%%%%%%%%%%%%%%%%%%
\noplagiarism % PLEASE DON'T DELETE THIS LINE!
%%%%%%%%%%%%%%%%%%%%
\begin{abstract}
  % \mfigcap{width = 0.85\textwidth}{figs/George-Boole}{George Boole}
  % \begin{center}{\fcolorbox{blue}{yellow!60}{\parbox{0.65\textwidth}{\large
  %   \begin{itemize}
  %     \item
  %   \end{itemize}}}}
  % \end{center}
\end{abstract}
%%%%%%%%%%%%%%%%%%%%
\beginrequired
%%%%%%%%%%%%%%%

%%%%%%%%%%%%%%%
\begin{problem}[\score{10 = 5 + 5}]
  考虑循环语句
  \[
    \forkw \;(S_{1}; B; S_{2})\; S_{3}
  \]
  \begin{enumerate}[(1)]
    \item 请基于布尔表达式与控制流语句回填翻译方案为 \forkw{} 语句设计回填方案。
    \item 请使用回填方案为以下代码片段生成中间代码
      \begin{align*}
        &\forkw\; (i = 0; i > 1000 \;\&\&\; i < 2000; i = i + 1) \\
          &\qquad \ifkw\; i == 1231\; \\
          &\qquad\qquad \printkw\; {\textsl{``Happy New Year!''}}
      \end{align*}
      {\bf 要求:} 请使用图示(如注释语法树等)展示产生式与相应规则的使用情况。
  \end{enumerate}
\end{problem}

\begin{solution}
  \begin{enumerate}[(1)]
    \item  相当于 $S_{1}M_{1}\textbf{while}\;(B)\; \{M_{2}S_{3}M_{3}S_{2}\}$
    \begin{table}[!htbp]
      \centering
      \begin{tabular}{|c|l|}
        \cline{1-2}
        产生式&语义规则\\    
        \hline
        % \multirow{2}{*}{$S \rightarrow \forkw \;(S_{1}; B; S_{2})\; S_{3}$} 
        $S \rightarrow \forkw \;(S_{1}M_{1}; B; S_{2})\; M_{2} S_{3}M_{3}$ 
        &  $backpatch(S_1.next,M_1.instr);$ \\ 
        &  $backpatch(S_3.next,M_3.instr);$ \\ 
        &  $backpatch(S_2.next,M_1.instr);$ \\ 
        &  $backpatch(B.truelist,M_2.instr);$ \\ 
        &  $S.next = B.falselist;$ \\ 
        &  $gen('goto',M_1.instr);$ \\ 
        
        \hline
      \end{tabular}
    \end{table}
\newpage
    \item  
        101:i=0\\
        102:if i>1000 goto 104\\
        103:goto \_\\
        104:if i<2000 goto 106\\
        105:goto \_\\
        106:if i==1225 goto 108\\
        107:goto 109\\
        108:print "Merry Christmas"\\
        109:t1 = i + 1\\
        110:i = t1\\
        111:goto 102\\
        ...\\
        注:
        $M_1$=102\\
        $M_2$=106\\
        $M_3$=109\\
        \_表示待回填内容\\
       图示如下:\\ 
       \fig{width = 0.70\textwidth}{photo}
  \end{enumerate}
\end{solution}
%%%%%%%%%%%%%%%

%%%%%%%%%%%%%%%%%%%%
% 如果没有需要订正的题目,可以把这部分删掉

% \begincorrection
%%%%%%%%%%%%%%%%%%%%

%%%%%%%%%%%%%%%%%%%%
% 如果没有反馈,可以把这部分删掉
% \beginfb

% 你可以写
% \begin{itemize}
%   \item 对课程及教师的建议与意见
%   \item 教材中不理解的内容
%   \item 希望深入了解的内容
%   \item $\cdots$
% \end{itemize}
%%%%%%%%%%%%%%%%%%%%
\end{document}