% hw7.tex

% !TEX program = xelatex
%%%%%%%%%%%%%%%%%%%%
% see http://mirrors.concertpass.com/tex-archive/macros/latex/contrib/tufte-latex/sample-handout.pdf
% for how to use tufte-handout
\documentclass[a4paper, justified]{tufte-handout}

\input{hw-preamble} % feel free to modify this file if you understand LaTeX well
%%%%%%%%%%%%%%%%%%%%
\title{编译原理作业 (7)}
\me{陈彦泽}{181250015}{}{}
\date{2020年12月24日}
%%%%%%%%%%%%%%%%%%%%
\begin{document}
\maketitle
%%%%%%%%%%%%%%%%%%%%
\noplagiarism % PLEASE DON'T DELETE THIS LINE!
%%%%%%%%%%%%%%%%%%%%
\begin{abstract}
  % \mfigcap{width = 0.85\textwidth}{figs/George-Boole}{George Boole}
  % \begin{center}{\fcolorbox{blue}{yellow!60}{\parbox{0.65\textwidth}{\large
  %   \begin{itemize}
  %     \item
  %   \end{itemize}}}}
  % \end{center}
\end{abstract}
%%%%%%%%%%%%%%%%%%%%
\beginrequired
%%%%%%%%%%%%%%%

%%%%%%%%%%%%%%%
\begin{problem}[\score{10 = 5 + 5}]
  考虑循环语句
  \[
    \forkw \;(S_{1}; B; S_{2})\; S_{3}
  \]
  \begin{enumerate}[(1)]
    \item 在控制流语句翻译方案(参考课件或龙书\teal{图 6-36})的基础上为 \forkw{} 语句添加语义规则;
    \item 为以下代码片段生成中间代码
      \begin{align*}
        &\forkw\; (i = 0; i > 1000 \;\&\&\; i < 2000; i = i + 1) \\
          &\qquad \ifkw\; i == 1225\; \\
          &\qquad\qquad \printkw\; {\textsl{``Merry Christmas''}}
      \end{align*}
      ({\it 注: 最好用图示的方式展示产生式与相应规则的使用情况})
  \end{enumerate}
\end{problem}

\begin{solution}
  \begin{enumerate}[(1)]
    \item  相当于 $S_{1}\textbf{while}\;(B)\; \{S_{3}S_{2}\}$
    \begin{table}[!htbp]
      \centering
      \begin{tabular}{|c|l|}
        \cline{1-2}
        产生式&语义规则\\    
        \hline
        % \multirow{2}{*}{$S \rightarrow \forkw \;(S_{1}; B; S_{2})\; S_{3}$} 
        $S \rightarrow \forkw \;(S_{1}; B; S_{2})\; S_{3}$ & $S_1.next = newlabal()$ \\ 
        % \cline{2-2} 
        &  $begin = newlabal()$ \\ 
        &  $B.true = newlabal()$ \\ 
        &  $B.false = S.next$ \\ 
        &  $S_3.next = newlabel()$ \\ 
        &  $S_2.next = begin$ \\ 
        &  $S.code = S_1.code \; || \; label(S_1.next) \; || \; label(begin)$ \\ 
        &  $ \qquad\qquad || \; B.code  \; || \; label(B.true)$ \\ 
        &  $ \qquad\qquad || \; S_3.code  \; || \; label(S_3.next) \; || \; S_2.code $ \\ 
        &  $ \qquad\qquad || \; gen('goto' \; begin)$ \\ 
        \hline
      \end{tabular}
    \end{table}

    \item  i=0\\
        LABEL label1:\\
        LABEL label2:\\
        if i>1000 goto label5\\
        goto S.next\\
        LABEL label5: \\
        if i<2000 goto label3\\
        goto S.next\\
        LABEL label3:\\
        if i==1225 goto label6\\
        goto label4\\
        LABEL label6\\
        print "Merry Christmas"\\
        LABEL label4\\
        t1 = i + 1\\
        i = t1\\
        goto label2\\
        LABEL S.next\\
        ...

       图示如下:\\ 
      \fig{width = 0.70\textwidth}{graph}
  \end{enumerate}
\end{solution}
%%%%%%%%%%%%%%%

%%%%%%%%%%%%%%%%%%%%
% 如果没有需要订正的题目,可以把这部分删掉

% \begincorrection
%%%%%%%%%%%%%%%%%%%%

%%%%%%%%%%%%%%%%%%%%
% 如果没有反馈,可以把这部分删掉
% \beginfb

% 你可以写
% \begin{itemize}
%   \item 对课程及教师的建议与意见
%   \item 教材中不理解的内容
%   \item 希望深入了解的内容
%   \item $\cdots$
% \end{itemize}
%%%%%%%%%%%%%%%%%%%%
\end{document}