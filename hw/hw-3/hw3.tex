% hw3.tex

% !TEX program = xelatex
%%%%%%%%%%%%%%%%%%%%
% see http://mirrors.concertpass.com/tex-archive/macros/latex/contrib/tufte-latex/sample-handout.pdf
% for how to use tufte-handout
\documentclass[a4paper, justified]{tufte-handout}

\input{hw-preamble} % feel free to modify this file if you understand LaTeX well
%%%%%%%%%%%%%%%%%%%%
\title{编译原理作业 (3)}
\me{陈彦泽}{181250015}{}{}
\date{2020年11月26日}
%%%%%%%%%%%%%%%%%%%%
\begin{document}
\maketitle
%%%%%%%%%%%%%%%%%%%%
\noplagiarism % PLEASE DON'T DELETE THIS LINE!
%%%%%%%%%%%%%%%%%%%%
\begin{abstract}
  % \mfigcap{width = 0.85\textwidth}{figs/George-Boole}{George Boole}
  % \begin{center}{\fcolorbox{blue}{yellow!60}{\parbox{0.65\textwidth}{\large
  %   \begin{itemize}
  %     \item
  %   \end{itemize}}}}
  % \end{center}
\end{abstract}
%%%%%%%%%%%%%%%%%%%%
\beginrequired
%%%%%%%%%%%%%%%

%%%%%%%%%%%%%%%
\begin{problem}[\score{10 = 2 + 2 + 2 + 1 + 3}]
  给定下述文法$G$,
  \begin{align}
    S &\to aAb \\[8pt]
    S &\to bAa \\[8pt]
    A &\to cS \\[8pt]
    A &\to \epsilon
  \end{align}

  \begin{enumerate}[(1)]
    \item 为该文法计算必要的\first{}集合;
    \item 为该文法计算必要的\follow{}集合;
    \item 为该文法设计预测分析表;
    \item 该文法是$LL(1)$文法吗? 请说明理由。
    \item 如果该文法是$LL(1)$文法, 请给出相应的 $LL(1)$语法分析器的伪代码
      (可以使用递归下降实现框架, 也可以使用非递归的版本);
      如果该文法不是$LL(1)$文法, 请将其改造成 $LL(1)$ 文法 (不必再重复各小题)。
  \end{enumerate}
\end{problem}

\begin{solution}

\begin{enumerate}[(1)]
    
  \item 
  $FISTR(S) = \{a,b\}$\\
    $FISTR(A) = \{c,\epsilon\}$
  \item 
    $FOLLOW(S) = \{a,b,\$\}$\\
    $FOLLOW(A) = \{a,b\}$
  \item 
    \begin{tabular}{|c|c|c|c|c|}% 通过添加 | 来表示是否需要绘制竖线
    \hline  % 在表格最上方绘制横线
    \ &a&b&c&\$\\
    \hline  %在第一行和第二行之间绘制横线
    S&1&2&\ &\ \\
    \hline % 在表格最下方绘制横线
    A&4&4&3&\ \\
    \hline % 在表格最下方绘制横线
    \end{tabular}
  \item 
    是,因为预测分析表无冲突
  \item 
  \begin{algorithm}
    \begin{algorithmic}[1]
      \Procedure{MATCH($t$)}{} 
        \If{token=t}
          \State token $\gets$ NEXT-TOKEN()
        \ElsIf{$peek\not=$'E'}
          \State ERROR(token,$t$)
        \EndIf
      \EndProcedure
    \end{algorithmic}
  \end{algorithm}
  \begin{algorithm}
    \begin{algorithmic}[1]
      \Procedure{$S()$}{} 
        \If{token='a'}
          \State MATCH('a')
          \State $A()$
          \State MATCH('b')
        \ElsIf{token='b'}
          \State MATCH('b')
          \State $A()$
          \State MATCH('a')
        \Else
          \State ERROR(token,\{'a','b'\})
        \EndIf
      \EndProcedure
    \end{algorithmic}
  \end{algorithm}
  \begin{algorithm}
    \begin{algorithmic}[1]
      \Procedure{$A()$}{} 
        \If{token in $\{'a','b'\}$}
          \State break;
        \ElsIf{token='c'}
          \State MATCH('c')
          \State $S()$
        \Else
          \State ERROR(token,\{'a','b','c'\})
        \EndIf
      \EndProcedure
    \end{algorithmic}
  \end{algorithm}
\end{enumerate}

\end{solution}
%%%%%%%%%%%%%%%

%%%%%%%%%%%%%%%%%%%%
% 如果没有需要订正的题目,可以把这部分删掉

% \begincorrection
%%%%%%%%%%%%%%%%%%%%

%%%%%%%%%%%%%%%%%%%%
% 如果没有反馈,可以把这部分删掉
% \beginfb

% 你可以写
% \begin{itemize}
%   \item 对课程及教师的建议与意见
%   \item 教材中不理解的内容
%   \item 希望深入了解的内容
%   \item $\cdots$
% \end{itemize}
%%%%%%%%%%%%%%%%%%%%
\end{document}