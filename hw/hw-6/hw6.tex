% hw6.tex

% !TEX program = xelatex
%%%%%%%%%%%%%%%%%%%%
% see http://mirrors.concertpass.com/tex-archive/macros/latex/contrib/tufte-latex/sample-handout.pdf
% for how to use tufte-handout
\documentclass[a4paper, justified]{tufte-handout}

\input{hw-preamble} % feel free to modify this file if you understand LaTeX well
%%%%%%%%%%%%%%%%%%%%
\title{编译原理作业 (6)}
\me{陈彦泽}{181250015}{}{}
\date{2020年12月17日}
%%%%%%%%%%%%%%%%%%%%
\begin{document}
\maketitle
%%%%%%%%%%%%%%%%%%%%
\noplagiarism % PLEASE DON'T DELETE THIS LINE!
%%%%%%%%%%%%%%%%%%%%
\begin{abstract}
  % \mfigcap{width = 0.85\textwidth}{figs/George-Boole}{George Boole}
  % \begin{center}{\fcolorbox{blue}{yellow!60}{\parbox{0.65\textwidth}{\large
  %   \begin{itemize}
  %     \item
  %   \end{itemize}}}}
  % \end{center}
\end{abstract}
%%%%%%%%%%%%%%%%%%%%
\beginrequired
%%%%%%%%%%%%%%%

%%%%%%%%%%%%%%%
\begin{problem}[\score{10 = 5 + 5}]
  以下文法 $G$ 描述了带可选小数部分的二进制数。
  我们希望通过属性文法计算二进制数对应的十进制表示。
  \begin{align*}
    N &\to L . L \\
    N &\to L \\
    L &\to LB \\
    L &\to B \\
    B &\to 1 \\
    B &\to 0
  \end{align*}

  \begin{enumerate}[(1)]
    \item 请给出一种 $S$ 属性翻译方案。

      {\it 提示: $123 = (1 \times 10 + 2) \times 10 + 3$}
    \item 请给出一种 $L$ 属性翻译方案。
  \end{enumerate}
\end{problem}
\newpage
\begin{solution}
  \begin{enumerate}[(1)]
    \item  $S$ 属性翻译方案,综合属性为val、len:\\
      \begin{align*}
        N &\to L_1 . L_2 & \{N.val = L_1.val + L_2.val \div 2^{L_2.len};\}\\
        N &\to L &\{N.val = L.val;\}\\
        L &\to L_1B &\{L.val = L_1.val \times  2 + B.val;L.len = L_1.len+1;\}\\
        L &\to B &\{L.val = B.val;L.len = 1;\}\\
        B &\to 1 &\{B.val = 1;\}\\
        B &\to 0 &\{B.val = 0;\}
      \end{align*}
    \item $L$ 属性翻译方案,综合属性为val、syn,继承属性为side、pos:\\
      \begin{align*}
        N \to &\{L_1.side = 1;L_1.pos=0;\}L_1 .\\
              &\{L_2.side = 0;L_2.pos=-1;\} L_2 \\
              &\{N.val = L_1.val + L_2.val\};\\
        N \to &\{L_1.side = 1;L_1.pos=0;\}L \\
              &\{N.val = L.val;\}\\
        L \to &\{L_1.pos=L.pos+L.side;L_1.side=L.side;\}L_1\\
              &\{L.syn=L_1.syn-1;B.pos=L.syn\}B \\
              &\{L.val = L_1.val + B.val;\}\\
        L \to &\{B.pos=L.pos\}B \\
              &\{L.val = B.val;L.syn = B.syn;\}\\
        B \to &1 \{B.val = 2^{B.pos};B.syn=B.pos;\}\\
        B \to &0 \{B.val = 0;B.syn=B.pos;\}
      \end{align*}
      举个例子,10.11翻译过程如下
      \fig{width = 0.80\textwidth}{example}
  \end{enumerate}
\end{solution}
%%%%%%%%%%%%%%%

%%%%%%%%%%%%%%%%%%%%
% 如果没有需要订正的题目,可以把这部分删掉

% \begincorrection
%%%%%%%%%%%%%%%%%%%%

%%%%%%%%%%%%%%%%%%%%
% 如果没有反馈,可以把这部分删掉
% \beginfb

% 你可以写
% \begin{itemize}
%   \item 对课程及教师的建议与意见
%   \item 教材中不理解的内容
%   \item 希望深入了解的内容
%   \item $\cdots$
% \end{itemize}
%%%%%%%%%%%%%%%%%%%%
\end{document}