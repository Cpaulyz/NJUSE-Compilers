% hw4.tex

% !TEX program = xelatex
%%%%%%%%%%%%%%%%%%%%
% see http://mirrors.concertpass.com/tex-archive/macros/latex/contrib/tufte-latex/sample-handout.pdf
% for how to use tufte-handout
\documentclass[a4paper, justified]{tufte-handout}

\input{hw-preamble} % feel free to modify this file if you understand LaTeX well
%%%%%%%%%%%%%%%%%%%%
\title{编译原理作业 (4)}
\me{陈彦泽}{181250015}{}{}
\date{2020年12月03日}
%%%%%%%%%%%%%%%%%%%%
\begin{document}
\maketitle
%%%%%%%%%%%%%%%%%%%%
\noplagiarism % PLEASE DON'T DELETE THIS LINE!
%%%%%%%%%%%%%%%%%%%%
\begin{abstract}
  % \mfigcap{width = 0.85\textwidth}{figs/George-Boole}{George Boole}
  % \begin{center}{\fcolorbox{blue}{yellow!60}{\parbox{0.65\textwidth}{\large
  %   \begin{itemize}
  %     \item
  %   \end{itemize}}}}
  % \end{center}
\end{abstract}
%%%%%%%%%%%%%%%%%%%%
\beginrequired
%%%%%%%%%%%%%%%

%%%%%%%%%%%%%%%
\begin{problem}[\score{10 = 1 + 2 + 2 + 3 + 2}]
  给定下述文法$G$,
  \begin{align}
    L &\to LP \\[8pt]
    L &\to P \\[8pt]
    P &\to (P) \\[8pt]
    P &\to ()
  \end{align}

  \begin{enumerate}[(1)]
    \item 简述$G$所对应的语言;
    \item 为 $G$ 构造 $LR(0)$ 自动机;

      注意: 为了尽量统一状态编号, 便于批改, 当计算\textsc{closure}时, 请按照文法编号大小顺序加入新项。
      当计算\textsc{goto}$(I, X)$时, 请按照$I$中项的出现顺序依次考虑可能的转移符号$X$。

      要求: 给出初始状态$I_{0}$的计算方法以及\textsc{goto}($I_{0}, ($)的计算方法。
    \item 为该文法设计$LR(0)$分析表; 该文法是$LR(0)$文法吗? 请说明理由。
    \item 为该文法设计$SLR(1)$分析表; 该文法是$SLR(1)$文法吗? 请说明理由。

      要求: 请说明归约的设置条件。
    \item 如果该文法是$SLR(1)$文法, 请给出识别输入串$(())()$时自动机所经历的状态(编号)。
  \end{enumerate}
\end{problem}

\newpage
\begin{solution}
  \begin{enumerate}[(1)]
    \item 
      一组或多组成对括号,一组成对括号的形式为(),(()),(((...))),但不包括一对括号内包含多个成对括号的情况,如(()())
    \item 
      如图所示:\\
      \fig{width = 0.80\textwidth}{LR0}
    \item 
      
      该文法是$LR(0)$文法,因为$LR(0)$分析表没有冲突
      \begin{table}[!htbp]
      \centering
      \begin{tabular}{|c|c|c|c|c|c|c|}
      \hline
      \multicolumn{1}{|c|}{ \multirow{2}*{状态} }& \multicolumn{3}{c|}{ACTION} &\multicolumn{2}{c|}{GOTO}\\
      \cline{2-6}
      \multicolumn{1}{|c|}{}&(&)&\$&L&P\\
      \hline
      \cline{1-6}
      0&$s_3$& & &$g_1$&$g_2$\\
      \cline{1-6}
      1&$s_3$& &acc& &$g_4$\\
      \cline{1-6}
      2&$r_2$&$r_2$&$r_2$& & \\
      \cline{1-6}
      3&$s_3$&$s_6$& & &$g_5$\\
      \cline{1-6}
      4&$r_1$&$r_1$&$r_1$& & \\
      \cline{1-6}
      5& &$s_7$& & & \\
      \cline{1-6}
      6&$r_4$&$r_4$&$r_4$& & \\
      \cline{1-6}
      7&$r_3$&$r_3$&$r_3$& & \\
      \hline
      \end{tabular}
      \end{table}

  \item 

    $FOLLOW(L)=\{(,\$\}$\\
    $FOLLOW(P)=\{(,),\$\}$\\
    \begin{table}[!htbp]
    \centering
    \begin{tabular}{|c|c|c|c|c|c|c|}
    \hline
    \multicolumn{1}{|c|}{ \multirow{2}*{状态} }& \multicolumn{3}{c|}{ACTION} &\multicolumn{2}{c|}{GOTO}\\
    \cline{2-6}
    \multicolumn{1}{|c|}{}&(&)&\$&L&P\\
    \hline
    \cline{1-6}
    0&$s_3$& & &$g_1$&$g_2$\\
    \cline{1-6}
    1&$s_3$& &acc& &$g_4$\\
    \cline{1-6}
    2&$r_2$& &$r_2$& & \\
    \cline{1-6}
    3&$s_3$&$s_6$& & &$g_5$\\
    \cline{1-6}
    4&$r_1$& &$r_1$& & \\
    \cline{1-6}
    5& &$s_7$& & & \\
    \cline{1-6}
    6&$r_4$&$r_4$&$r_4$& & \\
    \cline{1-6}
    7&$r_3$&$r_3$&$r_3$& & \\
    \hline
    \end{tabular}
    \end{table}
  
  \item 
    0 3 3 6 5 7 2 1 3 6 4 1 \\  
    $I_0$: next-token ( $s_3$ to $I_3$          stack:$(_3$\\
    $I_3$: next-token ( $s_3$ to $I_3$          stack:$(_3(_3$\\
    $I_3$: next-token ) $s_6$ to $I_6$          stack:$(_3(_3)_6$\\
    $I_6$: next-token ) $r_4$、$g_5$ to $I_5$   stack:$(_3P_5$\\
    $I_5$: next-token ) $s_7$ to $I_7$          stack:$(_3P_5)_7$\\
    $I_7$: next-token ( $r_3$、$g_2$ to $I_2$   stack:$P_2$\\
    $I_2$: next-token ( $r_2$、$g_1$ to $I_1$   stack:$L_1$\\
    $I_1$: next-token ( $s_3$ to $I_3$          stack:$L_1(_3$\\
    $I_3$: next-token ) $s_6$ to $I_6$          stack:$L_1(_3)_6$\\
    $I_6$: next-token \$ $r_4$、$g_4$ to $I_4$  stack:$L_1P_4$\\
    $I_4$: next-token \$ $r_1$、$g_1$ to $I_1$  stack:$L_1$\\
    $I_1$: next-token \$ $acc$\\

  \end{enumerate}
\end{solution}
%%%%%%%%%%%%%%%

%%%%%%%%%%%%%%%%%%%%
% 如果没有需要订正的题目,可以把这部分删掉

% \begincorrection
%%%%%%%%%%%%%%%%%%%%

%%%%%%%%%%%%%%%%%%%%
% 如果没有反馈,可以把这部分删掉
% \beginfb

% 你可以写
% \begin{itemize}
%   \item 对课程及教师的建议与意见
%   \item 教材中不理解的内容
%   \item 希望深入了解的内容
%   \item $\cdots$
% \end{itemize}
%%%%%%%%%%%%%%%%%%%%
\end{document}