% hw5.tex

% !TEX program = xelatex
%%%%%%%%%%%%%%%%%%%%
% see http://mirrors.concertpass.com/tex-archive/macros/latex/contrib/tufte-latex/sample-handout.pdf
% for how to use tufte-handout
\documentclass[a4paper, justified]{tufte-handout}

\input{hw-preamble} % feel free to modify this file if you understand LaTeX well
%%%%%%%%%%%%%%%%%%%%
\title{编译原理作业 (5)}
\me{陈彦泽}{181250015}{}{}
\date{2020年12月10日}
%%%%%%%%%%%%%%%%%%%%
\begin{document}
\maketitle
%%%%%%%%%%%%%%%%%%%%
\noplagiarism % PLEASE DON'T DELETE THIS LINE!
%%%%%%%%%%%%%%%%%%%%
\begin{abstract}
  % \mfigcap{width = 0.85\textwidth}{figs/George-Boole}{George Boole}
  % \begin{center}{\fcolorbox{blue}{yellow!60}{\parbox{0.65\textwidth}{\large
  %   \begin{itemize}
  %     \item
  %   \end{itemize}}}}
  % \end{center}
\end{abstract}
%%%%%%%%%%%%%%%%%%%%
\beginrequired
%%%%%%%%%%%%%%%

%%%%%%%%%%%%%%%
\begin{problem}[\score{10 = 1 + 4 + 2 + 3}]
  给定下述文法$G$,
  \begin{align}
    L &\to LP \\[8pt]
    L &\to P \\[8pt]
    P &\to (P) \\[8pt]
    P &\to ()
  \end{align}

  \begin{enumerate}[(1)]
    \item 为后面的小题计算必要的\first{}集合与\follow{}集合 (可以直接转抄上次作业);
    \item 为 $G$ 构造 $LR(1)$ 自动机;

      注意: 为了尽量统一状态编号, 便于批改, 当计算\textsc{closure}时, 请按照文法编号大小顺序加入新项。
      当计算\textsc{goto}$(I, X)$时, 请按照$I$中项的出现顺序依次考虑可能的转移符号$X$。

      要求: 给出初始状态$I_{0}$的计算方法以及\textsc{goto}($I_{0}, ($)的计算方法。
    \item 为该文法设计$LR(1)$分析表; 该文法是$LR(1)$文法吗? 请说明理由。

      要求: 请说明归约的设置条件。
    \item 为该文法设计$LALR(1)$分析表; 该文法是$LALR(1)$文法吗? 请说明理由。
  \end{enumerate}
\end{problem}

\newpage
\begin{solution}
  \begin{enumerate}[(1)]
    \item
      \first{}$(L)=\{(\}$\\
      \first{}$(P)=\{(\}$\\
      \follow{}$(L)=\{(,\$\}$\\
      \follow{}$(P)=\{(,),\$\}$\\
 
    \item
      计算方法:\\  
      $\begin{aligned}
        I_0 &= CLOSURE([L'→·L,\$]) \\
        &= CLOSURE([L'→·L,\$],[L→·LP,\$],[L→·P,\$]) \\
        &= CLOSURE([L'→·L,\$],[L→·LP,\$/(],[L→·P,\$/(]) \\
        &= \{[L'→·L,\$],[L→·LP,\$/(],[L→·P,\$/(],[P→·(P),\$/(],[P→·(),\$/(]\} \\
      \end{aligned}$
      \\\\\\
      $\begin{aligned}
        GOTO(I_0,() &= CLOSURE([P→(·P),\$/(],[P→(·),\$/(]) \\
        &= \{[P→(·P),\$/(],[P→(·),\$/(],[P→·(P),)],[P→·(),)]\} \\
      \end{aligned}$
      \\
      \fig{width = 0.80\textwidth}{lr1}
    \item 
      如图所示,是$LR(1)$文法\\
      规约条件:在合并相同$LR(0)$核心项后\\
      $[k:A→\alpha·,a]\in I_i \wedge A \neq S'\Rightarrow ACTION[i,a] = rk$\\
    \begin{table}[!htbp]
      \centering
      \begin{tabular}{|c|c|c|c|c|c|c|}
      \hline
      \multicolumn{1}{|c|}{ \multirow{2}*{状态} }& \multicolumn{3}{c|}{ACTION} &\multicolumn{2}{c|}{GOTO}\\
      \cline{2-6}
      \multicolumn{1}{|c|}{}&(&)&\$&L&P\\
      \hline
      \cline{1-6}
      0&$s_3$& & &$g_1$&$g_2$\\
      \cline{1-6}
      1&$s_3$& &acc& &$g_4$\\
      \cline{1-6}
      2&$r_2$& &$r_2$& & \\
      \cline{1-6}
      3&$s_7$&$s_6$& & &$g_5$\\
      \cline{1-6}
      4&$r_1$& &$r_1$& & \\
      \cline{1-6}
      5& &$s_8$& & & \\
      \cline{1-6}
      6&$r_4$& &$r_4$& & \\
      \cline{1-6}
      7&$s_7$&$s_{10}$& & &$g_9$\\
      \cline{1-6}
      8&$r_3$& &$r_3$& & \\
      \cline{1-6}
      9& &$s_{11}$& & & \\
      \cline{1-6}
      10& &$r_4$& & & \\
      \cline{1-6}
      11& &$r_3$& & & \\
      \hline
      \end{tabular}
      \end{table}
    \newpage
  \item 
    合并$I_3$与$I_7$、$I_5$与$I_9$、$I_6$与$I_{10}$、$I_8$与$I_{11}$\\
    如图所示,是$LALR(1)$文法\\
    \begin{table}[!htbp]
      \centering
      \begin{tabular}{|c|c|c|c|c|c|c|}
      \hline
      \multicolumn{1}{|c|}{ \multirow{2}*{状态} }& \multicolumn{3}{c|}{ACTION} &\multicolumn{2}{c|}{GOTO}\\
      \cline{2-6}
      \multicolumn{1}{|c|}{}&(&)&\$&L&P\\
      \hline
      \cline{1-6}
      0&$s_{37}$& & &$g_1$&$g_2$\\
      \cline{1-6}
      1&$s_{37}$& &acc& &$g_4$\\
      \cline{1-6}
      2&$r_2$& &$r_2$& & \\
      \cline{1-6}
      37&$s_{37}$&$s_{610}$& & &$g_{59}$\\
      \cline{1-6}
      4&$r_1$& &$r_1$& & \\
      \cline{1-6}
      59& &$s_{811}$& & & \\
      \cline{1-6}
      610&$r_4$&$r_4$&$r_4$& & \\
      \cline{1-6}
      811&$r_{3}$&$r_{3}$&$r_{3}$& & \\
      \hline
      \end{tabular}
      \end{table}

  \end{enumerate}
\end{solution}
%%%%%%%%%%%%%%%

%%%%%%%%%%%%%%%%%%%%
% 如果没有需要订正的题目,可以把这部分删掉

% \begincorrection
%%%%%%%%%%%%%%%%%%%%

%%%%%%%%%%%%%%%%%%%%
% % 如果没有反馈,可以把这部分删掉
% \beginfb

% 你可以写
% \begin{itemize}
%   \item 对课程及教师的建议与意见
%   \item 教材中不理解的内容
%   \item 希望深入了解的内容
%   \item $\cdots$
% \end{itemize}
%%%%%%%%%%%%%%%%%%%%
\end{document}